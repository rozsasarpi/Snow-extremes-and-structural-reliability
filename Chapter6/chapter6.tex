\chapter{Exceptional ground snow per Eurocode}
\label{cha:exceptional}

% **************************** Define Graphics Path **************************
\ifpdf
    \graphicspath{{Chapter6/Figs/Raster/}{Chapter6/Figs/PDF/}{Chapter6/Figs/}}
\else
    \graphicspath{{Chapter6/Figs/Vector/}{Chapter6/Figs/}}
\fi

% **************************** Chapter Abstract ******************************
\leftskip=1cm
\noindent
\emph{The concept of exceptional ground snow load was introduced to EN 1991-1-3 based on French precedent, however, no other CEN member country has had similar provision before. Curiously, it has no quantitative definition in the standard, although this ``new action'' is governing the design of many lightweight structures. In this chapter, we critically examine the concept of exceptional ground snow load and offer a physical, statistical and reliability analysis based treatment. The analyses show that the concept of exceptional snow load provided in the related Eurocode background document is not justified statistically, either meteorologically. Reliability based definition is proposed that is generalized to other exceptional loads.  In contrast with the current definition it is believed to be founded on sound meteorological, statistical, and reliability principles. Moreover, it allows quantitative treatment opposed to the prescribing nature of the current one. The range of parameters where exceptional ground snow load should be considered is identified.}

\leftskip=0pt\rightskip=0pt

\section{Problem statement and the state of the art}
% Mirek ~ short chapter or annex!?

The joint European research effort \citep{Sanpaolesi1998}, which introduced the concept of exceptional snow action defines it as:
\begin{quote}
	``Isolated and very infrequent snowfalls where the resulting snow load is significantly greater than the loads in the general body of snow load data and its inclusion in that data set distorts the statistical analysis.''
\end{quote}
  
Later the report provides a working definition of exceptional snow load:
\begin{quote}
	``If the ratio of the largest load value to the characteristic load determined without the inclusion of that value is greater than 1.5 then the largest load value shall be treated as an exceptional value.''
\end{quote}

This definition is intended to be used to identify the ``isolated and very infrequent snowfalls''. However, it says nothing about how the standardized exceptional snow load model should be constructed from the identified exceptional loads. Moreover, there is a salient shortcoming of the above definition as it does not refer to any observation length. With increasing observation length the definition will almost surely identify exceptional loads, even if the sampled distribution is deliberately selected to be non-extreme. Furthermore, the definition is only related to annual maxima, this is clear from the report that considers this sole approach.

As stated in the scope and limits of this thesis (Section~\ref{sec:scope}) we focus solely on ground snow load. Hence, the snow load is to be understood as ground snow load and not snow on the roof.


\section{Solution strategy}
and here I write more \dots

\section{Results and discussion}

%\section{Extension to other random variables}
\section{Application example}

\section{Summary and conclusions}

Initial calculations show that the definition given in Eurocode background document is inappropriate; there seems to be no meteorological justification for exceptional snow; reliability based definition is proposed

