% ******************************************************************************
% ****************************** Custom Margin *********************************

% Add `custommargin' in the document class options to use this section
% Set {innerside margin / outerside margin / topmargin / bottom margin}  and
% other page dimensions
\ifsetCustomMargin
  \RequirePackage[left=30mm,right=20mm,top=30mm,bottom=20mm]{geometry}
  \setFancyHdr % To apply fancy header after geometry package is loaded
\fi

% *****************************************************************************
% ******************* Fonts (like different typewriter fonts etc.)*************

% Add `customfont' in the document class option to use this section

\ifsetCustomFont
  % Set your custom font here and use `customfont' in options. Leave empty to
  % load computer modern font (default LaTeX font).
  %\RequirePackage{helvet}

	\usepackage{mathpazo} % add possibly `sc` and `osf` options
	%\usepackage[T1]{fontenc}
	%\usepackage[garamond]{mathdesign}
	%\usepackage{eulervm}
	
  % For use with XeLaTeX
  %  \setmainfont[
  %    Path              = ./libertine/opentype/,
  %    Extension         = .otf,
  %    UprightFont = LinLibertine_R,
  %    BoldFont = LinLibertine_RZ, % Linux Libertine O Regular Semibold
  %    ItalicFont = LinLibertine_RI,
  %    BoldItalicFont = LinLibertine_RZI, % Linux Libertine O Regular Semibold Italic
  %  ]
  %  {libertine}
  %  % load font from system font
  %  \newfontfamily\libertinesystemfont{Linux Libertine O}
\fi

\usepackage{textgreek}
\usepackage{phaistos}
% well looking approximate symbol for text
%\newcommand\textsim{\raise.17ex\hbox{$\scriptstyle\sim$}}

% *****************************************************************************
% **************************** Custom Packages ********************************

% ************************* Algorithms and Pseudocode **************************

%\usepackage{algpseudocode}

% ************************* Language **************************
\usepackage[hungarian,british]{babel}

% ********************Captions and Hyperreferencing / URL **********************

% Captions: This makes captions of figures use a boldfaced small font.
\RequirePackage[small,bf]{caption}

\RequirePackage[labelsep=space,tableposition=top]{caption}
\renewcommand{\figurename}{Fig.} %to support older versions of captions.sty

% to refer the chapter names
\usepackage{nameref}

% ********************************** Tables ************************************
\usepackage{booktabs} % For professional looking tables
\usepackage{multirow}

\usepackage{rotating} % for floating landscape tables

\usepackage{multicol}
%\usepackage{longtable}
%\usepackage{tabularx}


\usepackage[table]{xcolor} % for table coloring, !!load before tikz!!

\usepackage{threeparttable} % for notes in tables

% *************************** Graphics and figures *****************************

\usepackage{graphicx}

%\usepackage{rotating}
%\usepackage{wrapfig}

% Uncomment the following two lines to force Latex to place the figure.
% Use [H] when including graphics. Note 'H' instead of 'h'
%\usepackage{float}
%\restylefloat{figure}

% Subcaption package is also available in the sty folder you can use that by
% uncommenting the following line
% This is for people stuck with older versions of texlive
%\usepackage{sty/caption/subcaption}
\usepackage{subcaption}

% ................................. TiKZ .......................................
% for tikz graphics
\usepackage{tikz}
\usepackage{verbatim}
\usetikzlibrary{arrows}
%\usepackage{standalone} % to load separate TiKZ files


% *********************************** SI Units *********************************
\usepackage{siunitx} % use this package module for SI units


% ******************************* Line Spacing *********************************

% Choose linespacing as appropriate. Default is one-half line spacing as per the
% University guidelines

% \doublespacing
% \onehalfspacing
% \singlespacing

% ************************ Typesetting / General *******************************
 
% microtype is automatically loaded if pdflatex is used!

% ************************ Formatting / Footnote *******************************

% Don't break enumeration (etc.) across pages in an ugly manner (default 10000)
%\clubpenalty=500
%\widowpenalty=500

%\usepackage[perpage]{footmisc} %Range of footnote options
%\usepackage{enumerate}
\usepackage{enumitem}

%\usepackage{hyphenat}

% ************************ References / Hyperrefernces *************************

% coloring references
%\usepackage{color}
%\usepackage[backref=page]{hyperref}

\usepackage{color,hyperref}
%\usepackage{color}
%\usepackage{hyperref}

% ************************ Coloring *******************************************

\definecolor{darkblue}{rgb}{0.0, 0.0, 0.3}
\definecolor{cornflowerblue}{rgb}{0.392, 0.584, 0.9294}
\definecolor{midnightblue}{rgb} {0.1, 0.1, 0.44} %{1,0.41,0.70}
%\definecolor{midnightblue}{rgb} {0, 0, 0} %for print
\definecolor{orange}{rgb}{1.0, 0.5, 0.0}
\definecolor{lightgrey}{rgb}{0.9, 0.9, 0.9}

% *****************************************************************************
% *************************** Bibliography  and References ********************

%\usepackage{cleveref} %Referencing without need to explicitly state fig /table

% Add `custombib' in the document class option to use this section
\ifuseCustomBib
   %\RequirePackage[square, sort, numbers, authoryear]{natbib} % CustomBib
   
   % add custom coloring
   \usepackage{hyperref}
   \hypersetup{colorlinks,breaklinks,
               linkcolor = midnightblue,urlcolor = midnightblue,
               anchorcolor = midnightblue,citecolor = midnightblue}
   \RequirePackage[round, sort, authoryear]{natbib} % CustomBib
   
   \usepackage{hypernat}
   \usepackage{etoolbox}
   \makeatletter
   \patchcmd{\BR@backref}{\newblock}{\newblock(page~}{}{}
   \patchcmd{\BR@backref}{\par}{)\par}{}{}
   \makeatother
%   
%\renewcommand{\backrefxxx}[3]{(page \hyperlink{page.#1}{#1})}
   % sourced from : https://tex.stackexchange.com/questions/36307/formatting-back-references-in-bibliography

% If you would like to use biblatex for your reference management, as opposed to the default `natbibpackage` pass the option `custombib` in the document class. Comment out the previous line to make sure you don't load the natbib package. Uncomment the following lines and specify the location of references.bib file

%\RequirePackage[backend=biber, style=numeric-comp, citestyle=numeric, sorting=nty, natbib=true]{biblatex}
%\bibliography{References/references} %Location of references.bib only for biblatex

\fi

% changes the default name `Bibliography` -> `References'
\renewcommand{\bibname}{References}

% package for multiple bibliographies
%\usepackage{chapterbib}

% ******************************** Roman Pages *********************************
% The romanpages environment set the page numbering to lowercase roman one
% for the contents and figures lists. It also resets
% page-numbering for the remainder of the dissertation (arabic, starting at 1).

\newenvironment{romanpages}{
  \setcounter{page}{1}
  \renewcommand{\thepage}{\roman{page}}}
{\newpage\renewcommand{\thepage}{\arabic{page}}}


% ******************************************************************************
% ************************* User Defined Commands ******************************
% ******************************************************************************

% *********** To change the name of Table of Contents / LOF and LOT ************

%\renewcommand{\contentsname}{My Table of Contents}
%\renewcommand{\listfigurename}{My List of Figures}
%\renewcommand{\listtablename}{My List of Tables}


% ********************** TOC depth and numbering depth *************************

\setcounter{secnumdepth}{2}
\setcounter{tocdepth}{2}


% ******************************* Nomenclature *********************************

% for two column nomenclature
\renewcommand*{\nompreamble}{\begin{multicols}{2}}
\renewcommand*{\nompostamble}{\end{multicols}}
\setlength{\columnsep}{3em}

%\usepackage{nomencl}
%\makenomenclature
% To change the name of the Nomenclature section, uncomment the following line

%\renewcommand{\nomname}{Symbols}


% ********************************* Appendix ***********************************

% The default value of both \appendixtocname and \appendixpagename is `Appendices'. These names can all be changed via:

%\renewcommand{\appendixtocname}{List of appendices}
%\renewcommand{\appendixname}{Appndx}

% *********************** Configure Draft Mode **********************************

% Uncomment to disable figures in `draftmode'
%\setkeys{Gin}{draft=true}  % set draft to false to enable figures in `draft'

% These options are active only during the draft mode
% Default text is "Draft"
%\SetDraftText{DRAFT}

% Default Watermark location is top. Location (top/bottom)
%\SetDraftWMPosition{bottom}

% Draft Version - default is v1.0
%\SetDraftVersion{v1.1}

% Draft Text grayscale value (should be between 0-black and 1-white)
% Default value is 0.75
%\SetDraftGrayScale{0.8}


% ******************************** Todo Notes **********************************
%% Uncomment the following lines to have todonotes.

\ifsetDraft
	\usepackage[colorinlistoftodos]{todonotes}
	\newcommand{\mynote}[1]{\todo[author=kks32,size=\small,inline,color=green!40]{#1}}
\else
	\newcommand{\mynote}[1]{}
	\newcommand{\listoftodos}{}
\fi

% Example todo: \mynote{Hey! I have a note}
